\documentclass[10pt]{article}
\usepackage{graphicx}
\usepackage{geometry}
\usepackage[utf8]{inputenc}
\addtolength{\topmargin}{-10pt}
\addtolength{\textheight}{120pt}
\title{SecondDeliverable}
\author{singla.ankur147 }
\date{July 2019}

\begin{document}
\includegraphics[scale=0.2]{concordia}

\Large SOEN6011- Software Engineering Processes\\
\begin{center}
\textbf{Deliverable 2}\\
F3: sinh(x)
\end{center}

Github Link: https://github.com/SinglaAnkur/SOEN\_6011
\vspace{90mm}\\
\begin{large}
\textbf{Submitted by:}\\
\textbf{Ankur Singla}\\
\textbf{40090208}\\
\end{large}
\newpage
\tableofcontents
\newpage
\section{Problem 4}
\subsection{Debugger:}
Debugger makes it easy to pause the execution of a program at a certain point and inspect the code.[1] IntelliJ IDEA provides inbuilt Java Debugger. Below are the advantages and disadvantages .\\
\subsection*{\large 1.1.1 Advantages}
\begin{description}
\begin{itemize}
\item [1.1.1.1] IntelliJ IDEA is a good tool for application development. Inbuilt debugger saves the hassle of installing a separate software for debugging.
\item [1.1.1.2]Adding a breakpoint[2] is just a click of mouse. Breakpoint could be added to a statement or a method.
\item [1.1.1.3]IntelliJ Debugger works comparatively faster than other(Eclipse) debuggers.
\item [1.1.1.4]If a statement contains multiple method calls, then its "Smart Step Into"[3] feature allows to step into a particular method in the call.
\item [1.1.1.5]It allows to interact with the result of evaluating expression at any given instantanious time.
\end{itemize}
\end{description}
\subsection*{\large 1.1.2 Disadvantages}
\begin{description}
\begin{itemize}
\item [1.1.2.1]It encounters large RAM usage. Sometimes while debugging CPU usage spikes high. 
\end{itemize}
\end{description}
\newpage
\subsection{Checkstyle:}
Checkstyle is a tool used to analyze sourcce code in software development. It ensures that proper coding rules/standards are being followed.
Below are the advantages and disadvantages of using Checkstyle..\\
\subsection*{\large 1.2.1 Advantages}
\begin{description}
\begin{itemize}
\item [1.2.1.1] Checkstyle automates the process of source code check and enforces good programming practices.
\item [1.2.1.2]It reports possible different ways of coding which is not only useful for testing but for learning as well.
\item [1.2.1.3]It helps to improve the quality, readability and reusability of the code[4].
\item [1.2.1.4]It could be configured to support any coding standard.
\item [1.2.1.5]Provides support for JavaDocs.
\item [1.2.1.6]Easy to setup[5].
\end{itemize}
\end{description}
\subsection*{\large 1.1.2 Disadvantages}
\begin{description}
\begin{itemize}
\item [1.2.2.1]This tool cannot state if the code is correct or complete.
\item [1.2.2.2]Strict imposition of the coding conventions leads to difficulty in sharing code with outside parties unless they follow same set of standards.
\end{itemize}
\end{description}
\newpage
\section{References}
{[1]} https://www.tutorialspoint.com/intellij\_idea/intellij\_idea\_debugging.htm\\
{[2]} https://en.wikipedia.org/wiki/Breakpoint\\
{[3]} https://www.jetbrains.com/help/idea/debugging-code.html\\
{[4]} https://en.wikipedia.org/wiki/Checkstyle\\
{[5]} https://medium.com/@jayanga/how-to-configure-checkstyle-and-findbugs-plugins-to-intellij-idea-for-wso2-products-c5f4bbe9673a\\
\end{document}
