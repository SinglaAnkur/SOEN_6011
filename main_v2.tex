\documentclass{article}
\usepackage[utf8]{inputenc}
\usepackage{graphicx}
\usepackage{siunitx}

\usepackage[ruled,vlined]{algorithm2e}
\SetKwProg{Fn}{Function}{}{end}
\SetKwFunction{FnPow}{FnPower}%
\SetKwFunction{FnFact}{FnFactorial}%
\SetKwFunction{FnCalc}{FnCalculate}%
\SetKwFunction{FnExp}{FnExponential}%
\SetKwFunction{FnComp}{FnComparison}%

\begin{document}
\includegraphics[scale=0.2]{concordia}
\begin{small}
\Large SOEN6011- Software Engineering Processes\\
\linebreak
{F3: sinh(x)}\\
Github Link: https://github.com/SinglaAnkur/SOEN\_6011
\vspace{110mm}\\
\textbf{Submitted by:}\\
\textbf{Ankur Singla}\\
\textbf{40090208}\\
\end{small}
\newpage
\tableofcontents
\newpage
\section{Problem 1}
\subsection{Introduction}
\textbf{sinh(x)} is a hyperbolic[1] sine. This function is related to a hyperbola in the same way as the trigonometric function sin(x) is related to a circle.\\
Consider a hyperbola:  $x^2-y^2=1$\\

\begin{center}
\includegraphics[scale=0.9]{hyperbola.JPG}\\
\caption{Figure:1.1-1 Hyperbola}\\
\captionsource{Google Images}
\end{center}
\begin{small}
sinh(x) would be the length of perpendicular drawn from a vertex on hyperbola to the x-axis. The vertex is 1 unit far from the origin.[2]\\
\[sinh(x)=\frac {e^x-e^{-x}}{2}\]\\
Domain: $(-\infty,\infty)$\\
Co-domain: $(-\infty,\infty)$\\
\end{small}

\subsection{ Characteristics}
\begin{itemize}
\item [1.2.1]As x increases, $e^x$ increases quickly and $e^{−x}$ decreases quickly.[1]\\
$sinh(x) \approx \frac{e^x}{2}$
\item [1.2.2]If x decreases, $e^x$ decreases quickly and $-e^{−x}$ becomes large.[1]\\
$sinh(x) \approx \frac{- e^{-x}}{2}$
\item [1.2.3]sinh(x) is an odd function.[1]\\
$sinh(−x) = −\sinh(x)$
\item [1.2.4]sinh(x)=0 for x=0.[1]\\ sinh(x) $\to \infty$ when x $\to \infty$.\\ sinh(x) $\to -\infty$ when x $\to -\infty$.
\end{itemize}

\section{Problem 2}
\subsection{Requirements}
\begin{itemize}
\item [2.1.1]The value of x shall be a Real number[1].
\end{itemize}


\subsection{Assumptions}
\begin{itemize}
\item [2.2.1]x shall be an independent variable[3].
\item [2.2.2]e shall be the base of the natural log[3].
\end{itemize}
\pagebreak
\section{Problem 3}
\subsection{Algorithm 1}
This algorithm is based on the expansion of sinh(x) function in the form of Taylor series[4]. The function has been expanded upto $n^{th}$ term. The algorithm is divided into three functions namely: FnPower, FnFactorial, FnCalculate. \\Function FnPower is used to calculate the numerator of a term. The value of $result$ variable is returned to function FnCalculate. FnFactorial is used to calculate the denominator of a term. The value of $fact$ variable is returned to function FnCalculate. Function FnCalculate calculates aggregate value of each term and adds it to the variable $sum$ . Output of $sum$ is our desired value of the function .
\begin{algorithm}
\caption{ Calculate $sinh(x)= x+ \frac {x^3}{3!}+ \frac {x^5}{5!} + ... + \frac {x^{2n+1}}{(2n+1)!} + ...$ }
\SetAlgoLined

\Fn(\tcc*[h]{Function to calculate power}){\FnPow{x,n}}{
	\KwResult{Value of $x^{2n+1}$}
	\Begin{

	$result \leftarrow x$;
	\For{$i\leftarrow 1$ \KwTo $n-1$}{
		$result \leftarrow result\times i$
	}
	\nl\KwRet{$result$}
	}
}
\Fn(\tcc*[h]{Function to calculate factorial}){\FnFact{n}}{
	\KwResult{Value of (2n+1)!}
	\Begin{

	$fact \leftarrow 1$;
	\For{$i\leftarrow 1$ \KwTo $n$}{
		$fact \leftarrow fact \times i$
	}
	\nl\KwRet{$fact$}
	}
}
\Fn(\tcc*[h]{Function to calculate final value}){\FnCalc{n}}{
	\KwData{Value of $x, x \in \mathbb{R}$ }
	\KwResult{Value of sinh(x)}
	\Begin{

	$sum \leftarrow 0$;
	\For{$i\leftarrow 0$ \KwTo $70$}{
		$sum \leftarrow sum +\frac{\FnPow(x,((2 * i) + 1))}{\FnFact((2 \times i) + 1))}$
	}
	\nl\KwRet{$sum$}
	}
}
\end{algorithm}

\subsection{Algorithm 2}
\begin{algorithm}
\caption{ Calculate $sinh(x)=\frac {e^x-e^{-x}}{2}$}
\SetAlgoLined
\Fn(\tcc*[h]{Function to calculate exponential}){\FnExp{n,x}}{
	\KwResult{Value of $e^{x}$}
	\Begin{

	$sum \leftarrow 1$;
	\For{$i\leftarrow n-1$ \KwTo $1$}{
		$sum \leftarrow 1+ x\times \frac{sum}{i}$
	}
	\nl\KwRet{$sum$}
	}
}
\Fn(\tcc*[h]{Finds value of the exponential based on the input value of x}){\FnComp{x}}{
	\KwResult{Value of $sinh(x)$}
	\Begin{


	$output \leftarrow 0$;
	$abs\_x \leftarrow x$;
	\uIf{x==0}{\nl\KwRet{$0$}}
	\uElseIf{$x$\textless$0$}
	{
	    $abs\_x=x\times(-1)$
	}
	\uIf{abs\_xc$0$ \&\& abs\_x	$\leq15$ }{
	\For{$n\leftarrow 2$ \KwTo $1000$}{
		$value\_e^x \leftarrow \FnExp(n,abs\_x)$\\
		$value\_e^{-x} \leftarrow \FnExp(n,-abs\_x)$
	}
	output={$\frac{value\_e^x - value\_e^{-x}}{2}$}
	}
	\uElseIf{abs\_x\textgreater$15$ \&\& abs\_x	$\leq700$ }
	{
	\For{$n\leftarrow 2$ \KwTo $1000$}{
		$value\_e^x \leftarrow \FnExp(n,abs\_x)$\\
	}
	output={$\frac{value\_e^x}{2}$}
	}
	\uIf{x\textless0}{\nl\KwRet{$output\times(-1)$}}
	\Else{\nl\KwRet{$output$}}
	}
}
\end{algorithm}
\vspace{5mm}
This algorithm is based on the mathematical expression of $sinh(x)=\frac{e^x-e^{-x}}{2}$. 
Lets expand $e^x$.[5]\\
$e^x= 1+x+\frac{x^2}{2!}+\frac{x^3}{3!}+....+\frac{x^n}{n!}$\\
$e^x= 1+(\frac{x}{1})\times(1+ (\frac{x}{2})\times(1+(\frac{x}{3})(....)))$\\
Using this expansion we can calculate the value of $e^x$.\\
Also $e^{-x}$ could be calculated by inserting x as -x in the expansion.
FnExponential function calculates above expansion . 
FnComparison function calls the FnExponential function based on the value of x.
\pagebreak
\subsection{Comparison}
After analyzing both algorithms, Algorithm 2 was selected as the choice. Below are the advantages and disadvantages of choosing the particular algorithm.

\subsubsection{Advantages}
\vspace{1mm}
\textbf{Algorithm 1}
\begin{itemize}
\item [3.1.1.1]It is easy to understand.
\item [3.1.1.2]It is easy to maintain.
\end{itemize}
\vspace{1mm}
\textbf{Algorithm 2}
\begin{itemize}
\item [3.1.1.3]It provides more accurate values.\\Example: For x=100, Algorithm 1 outputs as 1.3439570961820604E43 and Algorithm 2 outputs as 1.344058570908067E43.
\item [3.1.1.4]It provides output for higher domain: x $\in$ (-700,700) as compared to Algorithm 1 x $\in$ (-150,150)
\end{itemize}
\subsubsection{Disadvantages}
\textbf{Algorithm 1}
\begin{itemize}
\item [3.1.2.1]It provides inaccurate values as stated in [3.1.1.4]
\item [3.1.2.2]It results in Infinity for $x\textgreater 150$ and -Infinity for $x\textless -150$.
\end{itemize}
\vspace{1mm}
\textbf{Algorithm 2}
\begin{itemize}
\item [3.1.2.4]It is hard to maintain.
\item [3.1.2.5]It is hard to understand.
\item [3.1.2.5]It results in Infinity for $x\textgreater 700$ and -Infinity for $x\textless -700$.
\end{itemize}
\pagebreak
\section{References}
[1] MathCentre, 'Hyperbolic functions. Available: 2006 http://www.mathcentre.ac.uk/resources/workbooks/mathcentre/hyperbolicfunctions.pdf [Accessed: 3-July-2019].\linebreak[4]
[2] LumenCollege 'Equations of Hyperbolas' https://courses.lumenlearning.com/waymakercollegealgebra/chapter/equations-of-hyperbolas/ [Accessed: 3-July-2019]\linebreak[4]
[3] Alcyone Systems 'Mathematics reference' Available: 2019 http://www.alcyone.com/max/reference/maths/hyperbolic.html [Accessed: 4-July-2019]\linebreak[4]
[4] UIC 'Stat 401: Introduction to Probability' Available: 2006 http://homepages.math.uic.edu/~jyang06/stat401/handouts/handout5.pdf [Accessed: 13-July-2019]\linebreak[4]
[5] GeeksForGeeks team	 'Efficient program to calculate $e^x$ ' Available: 2012 https://www.geeksforgeeks.org/program-to-efficiently-calculate-ex/ [Accessed: 17-July-2019]\linebreak[4]
\end{document}