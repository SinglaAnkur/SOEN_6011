\documentclass{article}
\usepackage[utf8]{inputenc}
\usepackage{graphicx}
\graphicspath{ {./} }


\begin{document}
\includegraphics[scale=0.2]{concordia}
\begin{small}
\Large SOEN6011- Software Engineering Processes\\
\linebreak
F3: sinh(x)\\
\vspace{110mm}\\
\textbf{Submitted by:}\\
\textbf{Ankur Singla}\\
\textbf{40090208}\\
\end{small}
\newpage
\tableofcontents
\newpage
\section{Problem 1:}
\subsection{Introduction:}
\textbf{sinh(x)} is a hyperbolic sine. This function is related to a hyperbola in the same way as the trigonometric function sin(x) is related to a circle.\\
Consider a hyperbola:  $x^2-y^2=1$\\

\begin{center}
\includegraphics[scale=0.9]{hyperbola.JPG}\\
\caption{Figure:1.1-1 Hyperbola}
\end{center}
\begin{small}
sinh(x) would be the length of perpendicular drawn from a vertex on hyperbola to the x-axis. The vertex is 1 unit far from the origin.\\
\[sinh(x)=\frac {e^x-e^{-x}}{2}\]\\
Domain: $(-\infty,\infty)$\\
Co-domain: $(-\infty,\infty)$\\
\end{small}

\subsection{ Characteristics:}
\begin{itemize}
\item [1.2.1]As x increases, $e^x$ increases quickly and $e^{−x}$ decreases quickly.\\
$sinh(x) \approx \frac{e^x}{2}$
\item [1.2.2]If x decreases, $e^x$ decreases quickly and $-e^{−x}$ becomes large.\\
$sinh(x) \approx \frac{- e^{-x}}{2}$
\item [1.2.3]sinh(x) is an odd function.\\
$sinh(−x) = −sinh(x)$
\item [1.2.4]sinh(x) is zero for x=0.\\ sinh(x) tends to infinity when x tends to infinity.\\ sinh(x) tends to minus infinity when x tends to minus infinity.
\end{itemize}

\section{Problem 2:}
\subsection{Requirements:}
\subsubsection{Functional Requirements:}
\begin{itemize}
\item [2.1.1.1]The program shall throw an error[2.2.5] when the value of x is input out of the specified range[2.2.1].
\item [2.1.1.2]The error shall be displayed on the screen to the user.
\item [2.1.1.3]The user shall be able to view the output after program is executed.
\item [2.1.1.4]The program shall exit after displaying the output.
\item [2.1.1.5]The program shall exit after throwing an error.
\end{itemize}
\subsubsection{Non Functional Requirements:}
\begin{itemize}
\item [2.1.2.1]The user shall be able to run the program on Windows 7, Windows 8, Windows 10, Mac OS X, Linux operating systems.
\item [2.1.2.2]The user shall install Java Standard Edition Development Kit of the specified versions J2SE 5.0, Java SE 6, Java SE 7, Java SE 8.
\item [2.1.2.3]The user shall be able to run the program through the specified Java Integrated Development Environments IntelliJ IDEA , NetBeans, Eclipse, JDeveloper.
\item [2.1.2.4]The user doesnot need an internet connection to run the program.
\end{itemize}

\subsection{Assumptions:}
\begin{itemize}
\item [2.2.1]The user shall input the value of x between -300 to 700.
\item [2.2.2]The value of x shall be integer, floating point.
\item [2.2.3]The value of x shall not contain alphabets, keywords, space, special characters.
\item [2.2.4]The user shall press enter key after keying in value of x to display the output.
\item [2.2.5]The program shall display OutOfMemory error to the user.
\end{itemize}
\end{document}


